\documentclass[]{scrartcl}

\usepackage{amsmath}
\usepackage{bussproofs}
\usepackage{color}

\EnableBpAbbreviations

%opening
\title{A Program Verification Engine for Imperative Languages}
\subtitle{Course: Program Verification}
\author{Giovanni Garufi, Fabian Thorand}
\date{\today}

\newcommand{\blue}[1]{\textcolor{blue}{#1}}

\newcommand{\HT}[3]{\blue{\{#1\}} #2 \blue{\{#3\}}}

\begin{document}

\maketitle

\section{Introduction}

In this report we present our tool for formally verifying programs written in
imperative languages. While testing already significantly decreases the chances
of having bugs in a program, it is often hard or event impossible to write test
cases that cover every execution path.

One way of making sure a program works as intended is to formally prove that its
behavior adheres to the specification.
Such a specification usually consists of a pre- and a post-condition.
The intention is, that the results of a program satisfy the post-condition,
given that the inputs to the program fulfil the pre-condition.
The Hoare calculus provides one mathematical foundation for conducting such
proofs based on pre- and post-conditions.
But of course, doing proofs by hand is error-prone and tedious even for
simple programs.

Ideally, there would be a tool automatically proving properties about programs.
Unfortunately, it is -- in general -- undecidable to check whether a given
program (represented as a Turing machine) satisfies a given property, due to the
Halting problem\footnote{as shown by Rice's theorem}.
The power of turing machines is usually introduced in programming languages
by some form of loops or general recursion.

Fortunately, this can be migitated by requiring the programmer to annotate loops
with invariants, guiding the verification tool in the right direction.
For some specific cases it might even be possible to automatically infer those
invariants.

Using a technique called \emph{Predicate Transformers} it is possible to infer
the precondition required for a program to satisfy its post-condition.

Upon that technique we built our verification engine that we describe in the
following sections.

\section{Predicate Transformer-Based Verification}
  \subsection{Basic Statements}
  \subsection{Loops}
  \subsection{Arrays}
  \subsection{Program Calls}

\section{Invariant Inference}

  \subsection{Fixpoint Iteration}

  \subsection{Loop Unrolling}

\section{Related Works}

\section{Conclusion}

\end{document}
